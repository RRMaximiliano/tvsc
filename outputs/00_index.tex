\documentclass[a4paper]{article}
%-------------------------------------------------------------------
% Document specifics
\usepackage{setspace}												% For spacing, e.g., double line spacing, line spread
\usepackage{enumitem}												% Controls layout of itemize, enumerate, description
\usepackage{footnote}												% Improve on LaTeX's footnote handling
%\usepackage[margin=1in]{geometry}									% Flexible and complete interface to document dimensions
\usepackage{fullpage}												% Set all page margins to 1.5cm
\usepackage{changepage}												% Commands to change the page layout in the middle of a document
\usepackage[table]{xcolor}											% Provides easy driver-independent access to several kinds of color 
\definecolor{airforceblue}{rgb}{0.36, 0.54, 0.66}
\definecolor{olivine}{rgb}{0.6, 0.73, 0.45}
\usepackage{colortbl}												% Allows rows and columns to be coloured
\usepackage{nameref}												% Defines a \nameref command, that makes reference to an object
\usepackage{titling}												% Provides control over the typesetting of the \maketitle command 
\usepackage[titletoc, title]{appendix}								% For appendices
%-------------------------------------------------------------------
% Tables and numbers
\usepackage[flushleft]{threeparttable}								% Tables with captions and notes with the same width
\usepackage{threeparttablex}										% Minor improvements over threeparttable
\usepackage{dcolumn}												% Align on the decimal point of numbers in tabular columns
\usepackage{multirow}												% Create tabular cells spanning multiple rows
\usepackage{longtable}												% Tables that continue to the next page
\usepackage{booktabs}												% Enhance quality of tables
\usepackage{tabularx}												% Tabulars with adjustable-width columns
\usepackage{mathrsfs, amsfonts, amssymb, amsmath}					% RSFS fonts in maths, Set of fonts for use in mathematics
%\usepackage{slashbox}												% Produces tabular cells with diagonal lines
\usepackage{siunitx}												% Set of tools for authors to typeset quantities in a consistent way
\usepackage{array}													% Extending the array and tabular environments
\usepackage[sc]{mathpazo}											% Fonts to typeset mathematics to match Palatino
%-------------------------------------------------------------------
% Floats in general
\usepackage{adjustbox}												% Provides several macros to adjust boxed content
\usepackage{graphicx}												% Provides a key-value interface for \includegraphics
\usepackage{float}													% Improves interface for floating objects
\usepackage[skip = 0pt]{caption}									% Customizes the captions in floating environments
\captionsetup{justification=centering}								% Captions setup
\usepackage{subcaption}												% Supports for sub-captions
\usepackage{rotating}												% Rotation tools, including rotated full-page floats
\usepackage{pdflscape}												% Makes landscape pages display as landscape
\usepackage{lscape}													% Place selected parts of a document in landscape
\usepackage{afterpage}												% Executes command after the next page break
\usepackage{tikz}													% For complex graphs in LaTeX
\usetikzlibrary{positioning}
%\tikzset{mynode/.style={draw,text width=1in,align=center}}
\tikzset{mynode/.style={draw,align=center}}
%\usepackage[nolists,heads]{endfloat}								% Moves floats to the end
%\usepackage[all]{hypcap}											% Adjusts the anchors of captions
%\usepackage{ctable}												% Flexible typesetting of table and figure floats using key/value directives
%-------------------------------------------------------------------
% Languages, encoding 
\usepackage[english]{babel}											% Multilingual support for Plain TeX or LaTeX
\usepackage[utf8]{inputenc}											% Accept different input encodings
\usepackage[autostyle, english = american]{csquotes}				% Provides advanced facilities for inline and display quotations
\usepackage[T1]{fontenc}											% Standard package for selecting font encodings
%-------------------------------------------------------------------
% MISC
\usepackage{lipsum}													% Easy access to the Lorem Ipsum dummy text.
\usepackage[plainpages=false,pdfpagelabels]{hyperref}				% Handles cross-referencing commands in LaTeX
\usepackage{url}													% Verbatim with URL-sensitive line breaks
\urlstyle{rm}
%-------------------------------------------------------------------
% References
%\usepackage{apacite}												% APA references thingy
\usepackage[natbib, style=apa]{biblatex}							% For better citation
%\addbibresource{.bib} 										% Replace this with your actual bibfile name
%-------------------------------------------------------------------
% Other document setups
%%% Misplaced noalign
\makeatletter\let\expandableinput\@@input\makeatother
%%% Quotes multilingual 
\MakeOuterQuote{"}
%%% Matrices
\setcounter{MaxMatrixCols}{10}
%%% New theorem environments
\newtheorem{acknowledgement}{Acknowledgement}
\newtheorem{algorithm}{Algorithm}
\newtheorem{axiom}{Axiom}
\newtheorem{case}{Case}
\newtheorem{claim}{Claim}
\newtheorem{conclusion}{Conclusion}
\newtheorem{condition}{Condition}
\newtheorem{conjecture}{Conjecture}
\newtheorem{corollary}{Corollary}
\newtheorem{criterion}{Criterion}
\newtheorem{definition}{Definition}
\newtheorem{example}{Example}
\newtheorem{exercise}{Exercise}
\newtheorem{lemma}{Lemma}
\newtheorem{notation}{Notation}
\newtheorem{problem}{Problem}
\newtheorem{proposition}{Proposition}
\newtheorem{remark}{Remark}
\newtheorem{solution}{Solution}
\newtheorem{assumption}{Assumption}
%%% New environment for landscape tables    
\newenvironment{ltable}{\begin{landscape}\begin{table}}{\end{table}\end{landscape}}
\newenvironment{ltablelong}{\begin{landscape}\begin{longtable}}{\end{longtable}\end{landscape}}
%%% New environment for column type (hide columns)
\newcolumntype{H}{>{\setbox0=\hbox\bgroup}c<{\egroup}@{}}
\newcolumntype{P}[1]{>{\centering\arraybackslash}p{#1}}
%%% Double perpendicular symbol (for independence)
\newcommand\independent{\protect\mathpalette{\protect\independenT}{\perp}}
\def\independenT#1#2{\mathrel{\rlap{$#1#2$}\mkern2mu{#1#2}}}
%%% For raw type of inputs (good for esttab command in Stata)
\makeatletter
\newcommand\primitiveinput[1]
{\@@input #1 }
\makeatother
%%% Numbering within sections (Tables)
%\numberwithin{table}{section}
%%% Define colors
\definecolor{darkblue}{rgb}{0,0,.4}
\definecolor{winered}{RGB}{198,13,37}
\hypersetup{colorlinks=true, 
			breaklinks=true, 
			citecolor=winered, 
			linkcolor=darkblue, 
			menucolor=darkblue, 
			urlcolor=darkblue}
%%% Return to section updates if available
\newcommand{\returnupdates}{%
Return to \nameref{sec:updates}.%
}
%%% Inserting Stata output (Stars)
\def\sym#1{\ifmmode^{#1}\else\(^{#1}\)\fi}

%-------------------------------------------------------------------
% Begin Document
%-------------------------------------------------------------------
\begin{document}

% Title, Author, Date
\title{\texttt{tvsc} command}
\author{Rony Rodrigo Maximiliano Rodriguez-Ramirez}
\date{}
\maketitle

%-------------------------------------------------------------------
% table 1
%-------------------------------------------------------------------
\newpage 

\begin{table}[H]
    \singlespacing
    \small
    \centering 
    \begin{adjustbox}{max width=\textwidth}
      \begin{threeparttable}
        \caption{Only Raw Differences}
        \label{tab:table1}
        \begin{tabular}[t]{@{}lccc}
            \toprule
            Variable & Treatment & Control & Differences \\
            \midrule 
            \addlinespace
            \primitiveinput{tables/t1.tex}
            \bottomrule
        \end{tabular}
        \begin{tablenotes}
            \setlength\labelsep{0pt}
            \footnotesize
            \item \textit{Notes}: Standard errors clustered at the region level.
        \end{tablenotes}
      \end{threeparttable}
   \end{adjustbox}
 \end{table}
%-------------------------------------------------------------------
% table 2
%-------------------------------------------------------------------
 \begin{table}[H]
    \singlespacing
    \small
    \centering 
    \begin{adjustbox}{max width=\textwidth}
      \begin{threeparttable}
        \caption{FE Differences}
        \label{tab:table2}
        \begin{tabular}[t]{@{}lccc}
            \toprule
            Variable & Treatment & Control & FE Differences \\
            \midrule 
            \addlinespace
            \primitiveinput{tables/t2.tex}
            \bottomrule
        \end{tabular}
        \begin{tablenotes}
            \setlength\labelsep{0pt}
            \footnotesize
            \item \textit{Notes}: Standard errors clustered at the region level.
        \end{tablenotes}
      \end{threeparttable}
   \end{adjustbox}
 \end{table}
%-------------------------------------------------------------------
% table 3
%-------------------------------------------------------------------
\begin{table}[H]
    \singlespacing
    \small
    \centering 
    \begin{adjustbox}{max width=\textwidth}
      \begin{threeparttable}
        \caption{Fe Differences}
        \label{tab:table3}
        \begin{tabular}[t]{@{}lcccc}
            \toprule
            Variable & Treatment & Control & Raw & FE  \\
            \midrule 
            \addlinespace
            \primitiveinput{tables/t3.tex}
            \bottomrule
        \end{tabular}
        \begin{tablenotes}
            \setlength\labelsep{0pt}
            \footnotesize
            \item \textit{Notes}: Standard errors clustered at the region level.
        \end{tablenotes}
      \end{threeparttable}
   \end{adjustbox}
 \end{table}
%-------------------------------------------------------------------
% table 4
%-------------------------------------------------------------------
\begin{table}[H]
  \singlespacing
  \small
  \centering 
  \begin{adjustbox}{max width=\textwidth}
    \begin{threeparttable}
      \caption{Raw Differences with observations}
      \label{tab:table4}
      \begin{tabular}[t]{@{}lcccc}
          \toprule
          Variable & Treatment & Control & Raw & Obs.  \\
          \midrule 
          \addlinespace
          \primitiveinput{tables/t4.tex}
          \bottomrule
      \end{tabular}
      \begin{tablenotes}
          \setlength\labelsep{0pt}
          \footnotesize
          \item \textit{Notes}: Standard errors clustered at the region level.
      \end{tablenotes}
    \end{threeparttable}
 \end{adjustbox}
\end{table}

%-------------------------------------------------------------------
% End document
%-------------------------------------------------------------------
\end{document}
%-------------------------------------------------------------------


